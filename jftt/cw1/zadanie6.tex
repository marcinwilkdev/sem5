\documentclass{article}
\usepackage[margin=1.25in]{geometry}
\usepackage{amsmath, amssymb, amsthm}
\usepackage{graphicx}
\usepackage[T1]{fontenc}
\usepackage[utf8]{inputenc}
\usepackage{lmodern}
\usepackage[skip=10pt]{parskip}

\begin{document}

\begin{center}
    \textbf{\LARGE Zadanie 6} \\
    {\large Marcin Wilk 261722} \\
\end{center}

\noindent \textbf{Treść zadania:} \\
Udowodnić lub obalić następujące tożsamości dla wyrażeń regularnych $r$, $s$ i $t$.

\[(rs + r)^*r = r(sr + r)^*\]

\begin{proof}[\unskip\nopunct]
\noindent \textbf{Dowód indukcyjny:} \\
Przeprowadzimy indukcję po liczbie powtórzeń wyrażenia regularnego w domknięciu Kleene'ego.

\[(rs + r)^nr = r(sr + r)^n\]

\noindent Dla $n = 0$:

\[(rs + r)^0r = r = r(sr + r)^0\]

\noindent Krok indukcyjny:
\begin{align*}
    (rs + r)^{n+1}r & = (rs + r)^n(rs + r)r \\
    & = (rs + r)^n(rsr + rr) \\
    & = (rs + r)^nr(sr + r) \\
    & = r(sr + r)^n(sr + r) \\
    & = r(sr + r)^{n + 1} \\
\end{align*}
\end{proof}

\[(r + s)^* = r^* + s^*\]

\begin{proof}[\unskip\nopunct]
\noindent \textbf{Dowód sprzeczności:} \\
Podamy przykład, w którym równość nie zachodzi co oznacza, że równanie nie jest prawdziwe.

\noindent Dla $r = a$, $s = b$:

\begin{center}
    $(a + b)^*$ akceptuje $ab$ \\[\baselineskip]
    $a^* + b^*$ nie akceptuje $ab$ \\[2\baselineskip]
\end{center}
\end{proof}

\pagebreak

\[s(rs + s)^*r = rr^*s(rr^*s)^*\]

\begin{proof}[\unskip\nopunct]
\noindent \textbf{Dowód sprzeczności:} \\
Podamy przykład, w którym równość nie zachodzi co oznacza, że równanie nie jest prawdziwe.

\noindent Dla $r = a$, $s = b$:

\begin{center}
    $b(ab + b)^*a$ akceptuje $ba$ \\[\baselineskip]
    $aa^*b(aa^*b)^*$ nie akceptuje $ba$
\end{center}
\end{proof}

\end{document}
