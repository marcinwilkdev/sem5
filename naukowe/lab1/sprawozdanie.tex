\documentclass{article}
\usepackage[margin=1.25in]{geometry}
\usepackage{amsmath, amssymb}
\usepackage{graphicx}
\usepackage[T1]{fontenc}
\usepackage[utf8]{inputenc}
\usepackage{lmodern}
\usepackage[skip=10pt]{parskip}

\begin{document}

\begin{center}
	\textbf{\LARGE Sprawozdanie lista 1} \\
	{\large Marcin Wilk 261722} \\

\end{center}
\noindent \textbf{\large Zadanie 1}

\noindent \textbf{Opis problemu} \\
Wyznaczyć iteracyjnie epsilon maszynowy, liczbę eta oraz maksymalną wartość dla typów zmiennopozycyjnych zgodnych
ze standardem IEEE 754 o długości half, single oraz double oraz porównać ich wartości z
wartościami wbudowanymi w język Julia oraz (dla epsilonu i wartości maksymalnej) język C.

\noindent \textbf{Opis rozwiązania} \\
Bierzemy wartość 1.0 i w każdej iteracji zmniejszamy ją dwukrotnie. W przypadku epsilonu \\
maszynowego kończymy iterowanie w momencie, w którym nasza wartość podzielona przez 2.0 i dodana do 1.0 jest równa
1.0, a w przypadku eta w momencie, w którym nasza wartość podzielona przez 2.0 jest równa 0.0. Otrzymane
wartości możemy porównać z wartościami wbudowanymi.

\noindent Dla wartości maksymalnej bierzemy wartość 1.0 i w każdej iteracji zwiększamy ją dwukrotnie. Pierwsze
iterowanie kończymy w momencie, w którym nasza wartość pomnożona przez 2.0 jest uznawana za nieskończoność.
Następnie wykonujemy drugie iterowanie, które jest binary searchem wyszukującym maxa pomiędzy
naszą wartością, a jej dwukrotną wielokrotnością.

\noindent \textbf{Rezultat}

\begin{center}
	\begin{tabular}{|l|l|l|l|}
		\hline
		\textbf{Typ} & \textbf{Eps (Iteracyjnie)}       & \textbf{Eps (Julia)}             & \textbf{Eps (C)}                 \\
		\hline
		Float16      & $0.000977$                       & $0.000977$                       & Nie dotyczy                      \\
		\hline
		Float32      & $1.1920929\cdot10^{-7}$          & $1.1920929\cdot10^{-7}$          & $1.1920929\cdot10^{-7}$          \\
		\hline
		Float64      & $2.220446049250313\cdot10^{-16}$ & $2.220446049250313\cdot10^{-16}$ & $2.220446049250313\cdot10^{-16}$ \\
		\hline
	\end{tabular}
\end{center}

\begin{center}
	\begin{tabular}{|l|l|l|}
		\hline
		\textbf{Typ} & \textbf{Eta (Iteracyjnie)} & \textbf{Eta (Julia)} \\
		\hline
		Float16      & $6.0\cdot10^{-8}$          & $6.0\cdot10^{-8}$    \\
		\hline
		Float32      & $1.0\cdot10^{-45}$         & $1.0\cdot10^{-45}$   \\
		\hline
		Float64      & $5.0\cdot10^{-324}$        & $5.0\cdot10^{-324}$  \\
		\hline
	\end{tabular}
\end{center}

\begin{center}
	\begin{tabular}{|l|l|l|l|}
		\hline
		\textbf{Typ} & \textbf{Max (Iteracyjnie)}        & \textbf{Max (Julia)}              & \textbf{Max (C)}                  \\
		\hline
		Float16      & $6.55\cdot10^{4}$                 & $6.55\cdot10^{4}$                 & Nie dotyczy                       \\
		\hline
		Float32      & $3.4028235\cdot10^{38}$           & $3.4028235\cdot10^{38}$           & $3.4028235\cdot10^{38}$           \\
		\hline
		Float64      & $1.7976931348623157\cdot10^{308}$ & $1.7976931348623157\cdot10^{308}$ & $1.7976931348623157\cdot10^{308}$ \\
		\hline
	\end{tabular}
\end{center}

\noindent \textbf{Wnioski} \\
Iteracyjne metody wyznaczania epsilonu, eta oraz wartości maksymalnej pokrywają się z wartościami
wbudowanymi w języki Julia oraz C co oznacza, że działają poprawnie.

\noindent Precyzja arytmetyki jest połową epsilonu maszynowego.

\noindent Liczba eta jest równa liczbie $MIN_{sub}$.

\noindent Funkcje \texttt{floatmin} z języka Julia zwracają najmniejsze wartości znormalizowane danego
typu float czyli są równe $MIN_{nor}$.

\pagebreak

\noindent \textbf{\large Zadanie 2}

\noindent \textbf{Opis problemu} \\
Sprawdzić czy wyrażenie $3(4/3 - 1) - 1$ wyznacza epsilon maszynowy dla typów zmiennopozycyjnych zgodnych
ze standardem IEEE 754 o długości half, single oraz double.

\noindent \textbf{Opis rozwiązania} \\
Przeprowadzamy kalkulację i porównujemy z wbudowaną wartością.

\noindent \textbf{Rezultat}

\begin{center}
	\begin{tabular}{|l|l|l|}
		\hline
		\textbf{Typ} & \textbf{Eta (Wykalkulowana)}     & \textbf{Eta (Julia)}             \\
		\hline
		Float16      & $0.000977$                       & $0.000977$                       \\
		\hline
		Float32      & $1.1920929\cdot10^{-7}$          & $1.1920929\cdot10^{-7}$          \\
		\hline
		Float64      & $2.220446049250313\cdot10^{-16}$ & $2.220446049250313\cdot10^{-16}$ \\
		\hline
	\end{tabular}
\end{center}

\noindent \textbf{Wnioski} \\
Dla wszystkich trzech typów zmiennopozycyjnych zgodnych ze standardem IEEE 754
działanie $3(4/3 - 1) - 1$ prawidłowo wyznacza epsilon maszynowy.

\noindent \textbf{\large Zadanie 3}

\noindent \textbf{Opis problemu} \\
Sprawdzić czy w arytmetyce \texttt{double} standardu IEEE 754 liczby zmiennopozycyjne są
równomiernie rozmieszczone w $[1,2]$ z krokiem $\delta=2^{-52}$.

\noindent \textbf{Opis rozwiązania} \\
Bierzemy wartość $1.0$ i w pętli dodajemy do niej $2^{-52}$ sprawdzając
czy nowa wartość jest równa kolejnej wartości \texttt{float} po poprzedniej wartości.
Iterowanie przerywamy w momencie, w którym nasza wartość wyniesie $2.0$. Jeżeli
wszystkie porównania w pętli się sprawdzą to zwracamy prawdę, w przeciwnym wypadku
zwracamy fałsz.

\noindent \textbf{Rezultat} \\
We wszystkich iteracjach algorytmu porównanie zwróciło prawdę co oznacza potwierdzenie
tezy postawionej w tym zadaniu.

\noindent \textbf{Wnioski} \\
Dla wszystkich liczb zmiennopozycyjnych w $[1,2]$ cecha musi być identyczna, ponieważ
odległość między kolejnymi liczbami jest identyczna. Przeprowadzając eksperyment dla innych
przedziałów możemy również pokazać gdzie cecha jest identyczna i jak rozmieszczone
są liczby na danym \\ przedziale. Dla przedziału $[\frac{1}{2}, 1]$ odległość między
kolejnymi liczbami wynosi $2^{-53}$, a dla przedziału $[2, 4]$ $2^{-51}$.

\pagebreak

\noindent \textbf{\large Zadanie 4}

\noindent \textbf{Opis problemu} \\
Znaleźć w arytmetyce \texttt{double} zgodnej z IEEE 754 liczbę zmiennopozycyjną
w zakresie $1 < x < 2$, taką że $x*(1/x) \neq 1$, a następnie znaleźć najmniejszą
taką liczbę.

\noindent \textbf{Opis rozwiązania} \\
Zaczynamy od liczby $1.0$ i iteracyjnie sprawdzamy kolejne liczby zmiennopozycyjne
czy sprawdzają założenie $x*(1/x) \neq 1$. Pierwsza znaleziona liczba będzie
jednocześnie najmniejszą liczbą w tym przedziale spełniającą to założenie.

\noindent \textbf{Rezultat}

\begin{center}
	\begin{tabular}{|c|}
		\hline
		\textbf{x}          \\
		\hline
		$1.000000057228997$ \\
		\hline
	\end{tabular}
\end{center}

\noindent \textbf{Wnioski} \\
Z powodu błędów zaokrągleń zmiennopozycyjnych nawet funkcje identycznościowe mogą
przestać dawać poprawne wyniki jeżeli przeprowadzają działania zmiennopozycyjne.

\noindent \textbf{\large Zadanie 5}

\noindent \textbf{Opis problemu} \\
Sprawdzić jak kolejność działań zmiennopozycyjnych w prostym algorytmie obliczającym iloczyn
skalarny dwóch wektorów wpływa na błąd rezultatu w arytmetyce \texttt{float} oraz \texttt{double}.

\noindent \textbf{Opis rozwiązania} \\
Przeprowadzić obliczenia kilkukrotnie z różną kolejnością działań zmiennopozycyjnych
i porównać wyniki z rezultatem faktycznym. Przy pierwszym podejściu sumujemy składowe
po kolei, przy drugim po kolei, ale od tyłu, a przy trzecim i czwartym odpowiednio posortowane
od największej do najmniejszej i od najmniejszej do największej składowej.

\noindent \textbf{Rezultat}

\noindent \textbf{Double}

\begin{center}
	\begin{tabular}{|l|l|l|}
		\hline
		\textbf{Faktyczny rezultat} & \textbf{Po kolei}                & \textbf{Po kolei (wstecz)}         \\
		\hline
		$-1.00657107\cdot10^{-11}$  & $1.0251881368296672\cdot10^{10}$ & $-1.5643308870494366\cdot10^{-10}$ \\
		\hline
	\end{tabular}
\end{center}

\begin{center}
	\begin{tabular}{|c|c|}
		\hline
		\textbf{Od największych} & \textbf{Od najmniejszych} \\
		\hline
		$0.0$                    & $0.0$                     \\
		\hline
	\end{tabular}
\end{center}

\noindent \textbf{Float}

\begin{center}
	\begin{tabular}{|l|l|l|}
		\hline
		\textbf{Faktyczny rezultat} & \textbf{Po kolei} & \textbf{Po kolei (wstecz)} \\
		\hline
		$-1.00657107\cdot10^{-11}$  & $-0.4999443$      & $-0.4543457$               \\
		\hline
	\end{tabular}
\end{center}

\begin{center}
	\begin{tabular}{|c|c|}
		\hline
		\textbf{Od największych} & \textbf{Od najmniejszych} \\
		\hline
		$-0.5$                   & $-0.25$                   \\
		\hline
	\end{tabular}
\end{center}

\pagebreak

\noindent \textbf{Wnioski} \\
Kolejność wykonywania działań w algorytmie zdecydowanie wpływa na to jak dokładny wynik dostaniemy,
co oznacza, że możemy manipulować tą kolejnością w celu otrzymania bardziej dokładnego wyniku.

\noindent \textbf{\large Zadanie 6}

\noindent \textbf{Opis problemu} \\
Porównać wyniki dwóch funkcji identycznych z matematycznego punktu widzenia, ale liczonych
w różny sposób w arytmetyce \texttt{double}:

\[ f(x) = \sqrt{x^2 + 1} - 1 \]
\[ g(x) = x^2 / ( \sqrt{x^2 + 1} + 1 ) \]

\noindent \textbf{Opis rozwiązania} \\
Napisać funkcje odopwiadające $f$ oraz $g$, a następnie porównać ich rezultaty dla kolejnych
ujemnych potęg liczby $8$.

\noindent \textbf{Rezultat}

\begin{center}
	\begin{tabular}{|c|c|c|}
		\hline
		\textbf{$x$} & \textbf{$f(x)$}                   & \textbf{$g(x)$}                   \\
		\hline
		$8^{-1}$     & $0.0077822185373186414$           & $0.0077822185373187065$           \\
		\hline
		$8^{-2}$     & $0.00012206286282867573$          & $0.00012206286282875901$          \\
		\hline
		$8^{-3}$     & $1.9073468138230965\cdot10^{-6}$  & $1.907346813826566\cdot10^{-6}$   \\
		\hline
		$8^{-4}$     & $2.9802321943606103\cdot10^{-8}$  & $2.9802321943606116\cdot10^{-8}$  \\
		\hline
		$8^{-5}$     & $4.656612873077393\cdot10^{-10}$  & $4.6566128719931904\cdot10^{-10}$ \\
		\hline
		$8^{-6}$     & $7.275957614183426\cdot10^{-12}$  & $7.275957614156956\cdot10^{-12}$  \\
		\hline
		$8^{-7}$     & $1.1368683772161603\cdot10^{-13}$ & $1.1368683772160957\cdot10^{-13}$ \\
		\hline
		$8^{-8}$     & $1.7763568394002505\cdot10^{-15}$ & $1.7763568394002489\cdot10^{-15}$ \\
		\hline
		$8^{-9}$     & $0.0$                             & $2.7755575615628914\cdot10^{-17}$ \\
		\hline
		$8^{-10}$    & $0.0$                             & $4.336808689942018\cdot10^{-19}$  \\
		\hline
	\end{tabular}
\end{center}

\noindent \textbf{Wnioski} \\
Obie funkcje dają poprawne wyniki, jednak funkcja $g$ daje poprawne wyniki dla mniejszych
liczb, w momencie w którym funkcja $f$ z powodu błedów zaokrąglenia daje wynik $0.0$
co oznacza, że funkcja $g$ jest bardziej wiarygodna.
\pagebreak


\noindent \textbf{\large Zadanie 7}

\noindent \textbf{Opis problemu} \\
Pochodne

\end{document}
