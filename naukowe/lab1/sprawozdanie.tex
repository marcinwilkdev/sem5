\documentclass{article}
\usepackage[margin=1.25in]{geometry}
\usepackage{amsmath, amssymb}
\usepackage{graphicx}
\usepackage[T1]{fontenc}
\usepackage[utf8]{inputenc}
\usepackage{lmodern}

\begin{document}

\begin{center}
    \textbf{\LARGE Sprawozdanie lista 1} \\
    {\large Marcin Wilk 261722} \\
\end{center}

\noindent \textbf{Zadanie 1} \\[\baselineskip]
\noindent \textbf{Opis problemu} \\
Wyznaczyć iteracyjnie epsilon maszynowy, liczbę eta oraz maksymalną wartość dla typów zmiennopozycyjnych zgodnych
ze standardem IEEE 754 o długości half, single oraz double oraz porównać ich wartości z
wartościami wbudowanymi w język Julia oraz (dla epsilonu i wartości maksymalnej) język C. \\

\noindent \textbf{Opis rozwiązania} \\
Bierzemy wartość 1.0 i w każdej iteracji zmniejszamy ją dwukrotnie. W przypadku epsilonu maszynowego
kończymy iterowanie w momencie, w którym nasza wartość podzielona przez 2.0 i dodana do 1.0 jest równa
1.0, a w przypadku eta w momencie, w którym nasza wartość podzielona przez 2.0 jest równa 0.0. Otrzymane
wartości możemy porównać z wartościami wbudowanymi. \\

\noindent Dla wartości maksymalnej bierzemy wartość 1.0 i w każdej iteracji zwiększamy ją dwukrotnie. Pierwsze
iterowanie kończymy w momencie, w którym nasza wartość pomnożona przez 2.0 jest uznawana za nieskończoność.
Następnie wykonujemy drugie iterowanie, które jest binary searchem wyszukującym maxa pomiędzy
naszą wartością, a jej dwukrotną wielokrotnością. \\

\noindent \textbf{Rezultat}

\begin{center}
\begin{tabular}{|l|l|l|l|}
    \hline
    \textbf{Typ} & \textbf{Eps (Iteracyjnie)} & \textbf{Eps (Julia)} & \textbf{Eps (C)} \\
    \hline
    Float16 & $0.000977$ & $0.000977$ & Nie dotyczy \\ 
    \hline
    Float32 & $1.1920929\cdot10^{-7}$ & $1.1920929\cdot10^{-7}$ & $1.1920929\cdot10^{-7}$ \\ 
    \hline
    Float64 & $2.220446049250313\cdot10^{-16}$ & $2.220446049250313\cdot10^{-16}$ & $2.220446049250313\cdot10^{-16}$ \\ 
    \hline
\end{tabular}
\end{center}

\begin{center}
\begin{tabular}{|l|l|l|}
    \hline
    \textbf{Typ} & \textbf{Eta (Iteracyjnie)} & \textbf{Eta (Julia)} \\
    \hline
    Float16 & $6.0\cdot10^{-8}$ & $6.0\cdot10^{-8}$ \\ 
    \hline
    Float32 & $1.0\cdot10^{-45}$ & $1.0\cdot10^{-45}$ \\ 
    \hline
    Float64 & $5.0\cdot10^{-324}$ & $5.0\cdot10^{-324}$ \\ 
    \hline
\end{tabular}
\end{center}

\begin{center}
\begin{tabular}{|l|l|l|l|}
    \hline
    \textbf{Typ} & \textbf{Max (Iteracyjnie)} & \textbf{Max (Julia)} & \textbf{Max (C)} \\
    \hline
    Float16 & $6.55\cdot10^{4}$ & $6.55\cdot10^{4}$ & Nie dotyczy \\ 
    \hline
    Float32 & $3.4028235\cdot10^{38}$ & $3.4028235\cdot10^{38}$ & $3.4028235\cdot10^{38}$ \\ 
    \hline
    Float64 & $1.7976931348623157\cdot10^{308}$ & $1.7976931348623157\cdot10^{308}$ & $1.7976931348623157\cdot10^{308}$ \\ 
    \hline
\end{tabular}
\end{center}

\noindent \textbf{Wnioski} \\
Iteracyjne metody wyznaczania epsilonu, eta oraz wartości maksymalnej pokrywają się z wartościami
wbudowanymi w języki Julia oraz C co oznacza, że działają poprawnie. \\

\noindent Precyzja arytmetyki jest połową epsilonu maszynowego. \\

\noindent Liczba eta jest równa liczbie $MIN_{sub}$. \\

\noindent Funkcje \texttt{floatmin} z języka Julia zwracają najmniejsze wartości znormalizowane danego
typu float czyli są równe $MIN_{nor}$. \\
\end{document}
