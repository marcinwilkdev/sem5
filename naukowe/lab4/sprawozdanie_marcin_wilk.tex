\documentclass{article}
\usepackage[margin=1.25in]{geometry}
\usepackage{amsmath, amssymb}
\usepackage{graphicx}
\usepackage[T1]{fontenc}
\usepackage[utf8]{inputenc}
\usepackage{lmodern}
\usepackage[skip=10pt]{parskip}
\usepackage{algpseudocode}

\begin{document}

\begin{center}
	\textbf{\LARGE Sprawozdanie lista 3} \\
	{\large Marcin Wilk 261722} \\

\end{center}

\noindent \textbf{\large Zadanie 1}

\noindent \textbf{Opis problemu} \\
Napisać funkcję obliczającą ilorazy różnicowe przy pomocy podanych
punktów $(x)$ oraz wartości funkcji w tych punktach $(f)$.

\noindent \textbf{Opis rozwiązania} \\
Zaimplementowanie algorytmu i przetestowanie na przykładowych funkcjach.

\noindent \textbf{Pseudokod}

\begin{algorithmic}
	\State \textbf{Input:} $x, f$
	\State $fCopy \gets f$
	\State $result \gets [fCopy[1]]$

	\For{$i $ \textbf{in} $1$ \textbf{to} $length(x)-1$}

	\For{$j $ \textbf{in} $1$ \textbf{to} $length(x)-i$}

	\State $fCopy[j] \gets (fCopy[j+1] - fCopy[j]) / (x[j+i] - x[j])$

	$result.push(fCopy[1])$

	\EndFor

	\State \Return{$result$}

	\EndFor
\end{algorithmic}

\noindent \textbf{\large Zadanie 2}

\noindent \textbf{Opis problemu} \\
Napisać funkcję obliczającą wartość wielomianu interpolacyjnego przy pomocy
podanych punktów $(x)$ oraz wartości ilorazów różnicowych w tych punktach $(fx)$
dla podanego punktu $(t)$.

\noindent \textbf{Opis rozwiązania} \\
Zaimplementowanie algorytmu i przetestowanie na przykładowych funkcjach.

\noindent \textbf{Pseudokod}
\begin{algorithmic}
	\State \textbf{Input:} $x, fx, t$
	\State $w \gets zeros(len(x))$
	\State $w[length(w)] \gets fx[length(fx)]$

	\For{$k $ \textbf{in} $length(x)-1$ \textbf{downto} $1$}
	\State $w[i] \gets w[i+1] * (t - x[i]) + fx[i]$
	\EndFor

	\State \Return{$w[1]$}

\end{algorithmic}

\pagebreak

\noindent \textbf{\large Zadanie 3}

\noindent \textbf{Opis problemu} \\
Napisać funkcję obliczającą współczynniki postaci naturalnej wielomianu
przy pomocy ilorazów różnicowych.

\noindent \textbf{Opis rozwiązania} \\
Zaimplementowanie algorytmu i przetestowanie na przykładowych funkcjach.

\noindent \textbf{Pseudokod}
\begin{algorithmic}
	\State $a \gets [fx[end]]$

	\For{$i $ \textbf{in} $length(x)-1$ \textbf{downto} $1$}

	\State $pushfront(a, 0.0)$

	\For{$j $ \textbf{in} $1$ \textbf{to} $length(a)-1$}

	\State $a[j] \gets a[j] - (x[i] * a[j+1])$

	\EndFor
	\State $a[1] \gets a[1] + fx[i]$

	\EndFor

	\State \Return{$a$}
\end{algorithmic}

\noindent \textbf{\large Zadanie 4}

\noindent \textbf{Opis problemu} \\
Napisać funkcję interpolującą zadana funkcję na danym przedziale,
która następnie plotuje zadaną funkcję oraz zinterpolowany wielomian.

\noindent \textbf{Opis rozwiązania} \\
Zaimplementować funkcję przy pomocy poprzednio zaimplemenotwanych algorytmów
oraz użyć wbudowane biblioteki do plotowania.

\noindent \textbf{Pseudokod}
\begin{algorithmic}
	\State \textbf{Input:} $f, a, b, n$
    \State $h \gets (b - a) / n$
    \State $xs \gets [a + k * h for k in 0:n]$
    \State $ys \gets [f(x) for x in xs]$
    \State $ilorazy \gets ilorazyRoznicowe(xs, ys)$

    \State $n *\gets 100$
    \State $h \gets (b - a) / n$

    \State $printxs \gets [a + k * h for k in 0:n]$
    \State $printys \gets [f(x) for x in printxs]$
    \State $polynomialValues \gets [warNewton(xs, ilorazy, x) for x in printxs]$

    \State $plot(printxs, [printys, polynomialValues])$
\end{algorithmic}

\noindent \textbf{\large Zadanie 5}

\noindent \textbf{Opis problemu} \\
Przetestować funkcję z poprzedniego zadania na kilku przykładach.

\noindent \textbf{Opis rozwiązania} \\
Uruchomić funkcję używając jako danych wejściowych:

a) $e^x$, $[0,1]$, $n=5,10,15$

b) $x^2sin(x)$, $[-1,1]$, $n=5,10,15$

\noindent \textbf{Rezultat}

\begin{center}
	\begin{tabular}{|c|c|c|c|}
		\hline
		\textbf{Pierwiastek} & \textbf{Wyliczone $x$} & \textbf{Wyliczone $f(x)$}    & \textbf{Liczba iteracji} \\
		\hline
		$>1$                 & $1.51220703125$        & $2.4548768085885797*10^{5}$  & $10$                     \\
		\hline
		$<1$                 & $0.619140625$          & $-4.8812505882511736*10^{5}$ & $8$                      \\
		\hline
	\end{tabular}
\end{center}

\noindent \textbf{Wnioski} \\
Jeżeli dobrze przekształcimy funcję, której rozwiązania chcemy wyznaczyć

\pagebreak

\noindent \textbf{\large Zadanie 6}

\noindent \textbf{Opis problemu} \\
Przetestować funkcję z poprzedniego zadania na kilku przykładach żeby
zaprezentować zjawisko rozbieżności.

\noindent \textbf{Opis rozwiązania} \\
Zastosować algorytmy do odszukania pierwiastków i porównać wyniki.

\noindent \textbf{Rezultat}

\noindent \textbf{$f_1(x) = e^{1-x} - 1$}

\begin{center}
	\begin{tabular}{|c|c|c|c|}
		\hline
		\textbf{Algorytm} & \textbf{Wyliczone $x$} & \textbf{Wyliczone $f(x)$}   & \textbf{Liczba iteracji} \\
		\hline
		Bisekcji          & $1.0000038146972656$   & $-3.814689989667386*10^{6}$ & $17$                     \\
		\hline
		Newtona           & $0.9999984358892101$   & $1.5641120130194253*10^{6}$ & $4$                      \\
		\hline
		Siecznych         & $1.000006725645982$    & $-6.725623364789435*10^{6}$ & $16$                     \\
		\hline
	\end{tabular}
\end{center}

\noindent \textbf{$f_2(x) = xe^{-x}$}

\begin{center}
	\begin{tabular}{|c|c|c|c|}
		\hline
		\textbf{Algorytm} & \textbf{Wyliczone $x$}      & \textbf{Wyliczone $f(x)$}  & \textbf{Liczba iteracji} \\
		\hline
		Bisekcji          & $7.62939453125*10^{6}$      & $7.62933632381113*10^{6}$  & $16$                     \\
		\hline
		Newtona           & $14.398662765680003$        & $8.03641534421721*10^{6}$  & $10$                     \\
		\hline
		Siecznych         & $7.2229790848421516*10^{6}$ & $7.222926913603708*10^{6}$ & $31$                     \\
		\hline
	\end{tabular}
\end{center}

\noindent \textbf{Wnioski} \\
W przypadku drugiej funkcji sytuacja wygląda podobnie do poprzedniego zadania, metoda Newtona daje
najszybsze rezultaty, a siecznych najwolniejsze, jednak dla pierwszej funkcji metoda bisekcji jest najwolniejsza.
W tym przypadku wpływ na to może mieć wybór odcinków początkowych nieprzychylnych dla danych metod.

\begin{center}
	\begin{tabular}{|c|c|c|c|}
		\hline
		\textbf{Wnioski} & \textbf{Wyliczone $x$} & \textbf{Wyliczone $f(x)$}   & \textbf{Liczba iteracji} \\
		\hline
		Pierwszy         & $0.9999999810061002$   & $1.8993900008368314*10^{8}$ & $5$                      \\
		\hline
		Drugi            & $14.380524159896261$   & $8.173205649825554*10^{6}$  & $4$                      \\
		\hline
		Trzeci           & $pf(x_0) = 0.0$        & $pf(x_0) = 0.0$             & $pf(x_0) = 0.0$          \\
		\hline
	\end{tabular}
\end{center}

Metoda Newtona radzi sobie z różnymi wybranymi $x_0$ dla obu funkcji, chyba, że jest to równe $0$
(wtedy wartość pochodnej funkcji za bardzo się zbliża do zera).

\end{document}
