\documentclass{article}
\usepackage[margin=1.25in]{geometry}
\usepackage{amsmath, amssymb}
\usepackage{graphicx}
\usepackage[T1]{fontenc}
\usepackage[utf8]{inputenc}
\usepackage{lmodern}
\usepackage[skip=10pt]{parskip}
\usepackage{algpseudocode}

\begin{document}

\begin{center}
	\textbf{\LARGE Sprawozdanie lista 4} \\
	{\large Marcin Wilk 261722} \\

\end{center}

\noindent \textbf{\large Zadanie 1}

\noindent \textbf{Opis problemu} \\
Napisać funkcję obliczającą ilorazy różnicowe przy pomocy podanych
punktów $(x)$ oraz wartości funkcji w tych punktach $(f)$.

\noindent \textbf{Opis rozwiązania} \\
Zaimplementowanie algorytmu i przetestowanie na przykładowych funkcjach.
Algorytm tworzy po kolei wszystkie ilorazy różnicowe każdego stopnia
(od $1$ do $n$), przy pomocy wyników z poprzedniego stopnia i pierwszy wynik
z każdego stopnia dodaje do rezultatu.

\noindent \textbf{Pseudokod}

\begin{algorithmic}
	\State \textbf{Input:} $x, f$
	\State $fCopy \gets f$
	\State $result \gets [fCopy[1]]$

	\For{$i $ \textbf{in} $1$ \textbf{to} $length(x)-1$}

	\For{$j $ \textbf{in} $1$ \textbf{to} $length(x)-i$}

	\State $fCopy[j] \gets (fCopy[j+1] - fCopy[j]) / (x[j+i] - x[j])$

	$result.push(fCopy[1])$

	\EndFor

	\State \Return{$result$}

	\EndFor
\end{algorithmic}

\noindent \textbf{\large Zadanie 2}

\noindent \textbf{Opis problemu} \\
Napisać funkcję obliczającą wartość wielomianu interpolacyjnego przy pomocy
podanych punktów $(x)$ oraz wartości ilorazów różnicowych w tych punktach $(fx)$
dla podanego punktu $(t)$.

\noindent \textbf{Opis rozwiązania} \\
Zaimplementowanie algorytmu i przetestowanie na przykładowych funkcjach.
Algorytm iteracyjnie powiększa wielomian w każdym kroku biorąc kolejne od
końca ilorazy różnicowe i podkładając je do wzoru. Na samym końcu otrzyma kompletny
wielomian więc wylicza jego wartość w zadanym punkcie.

\noindent \textbf{Pseudokod}
\begin{algorithmic}
	\State \textbf{Input:} $x, fx, t$
	\State $w \gets zeros(len(x))$
	\State $w[length(w)] \gets fx[length(fx)]$

	\For{$k $ \textbf{in} $length(x)-1$ \textbf{downto} $1$}
	\State $w[i] \gets w[i+1] * (t - x[i]) + fx[i]$
	\EndFor

	\State \Return{$w[1]$}

\end{algorithmic}

\pagebreak

\noindent \textbf{\large Zadanie 3}

\noindent \textbf{Opis problemu} \\
Napisać funkcję obliczającą współczynniki postaci naturalnej wielomianu
przy pomocy podanych punktów $(x)$ oraz ilorazów różnicowych $(fx)$.

\noindent \textbf{Opis rozwiązania} \\
Zaimplementowanie algorytmu i przetestowanie na przykładowych funkcjach.
Algorytm wykorzystuje metodę z poprzedniego zadania do wygenerowania tablicy
współczynników naturalnej postaci wielomianu. Robi to poprzez przesunięcie
współczynników w prawo, a następnie odjęcie współczynników pomnożonych przez
wartości punktów. Nastepnie dodaje kolejny iloraz różnicowy do pierwszego
współczynniku.

\noindent \textbf{Pseudokod}
\begin{algorithmic}
	\State \textbf{Input:} $x, fx$
	\State $a \gets [fx[end]]$

	\For{$i $ \textbf{in} $length(x)-1$ \textbf{downto} $1$}

	\State $pushfront(a, 0.0)$

	\For{$j $ \textbf{in} $1$ \textbf{to} $length(a)-1$}

	\State $a[j] \gets a[j] - (x[i] * a[j+1])$

	\EndFor
	\State $a[1] \gets a[1] + fx[i]$

	\EndFor

	\State \Return{$a$}
\end{algorithmic}

\noindent \textbf{\large Zadanie 4}

\noindent \textbf{Opis problemu} \\
Napisać funkcję interpolującą zadana funkcję $(f)$ na danym przedziale $(a, b)$,
podzielonym $(n)$ razy, która następnie plotuje zadaną funkcję oraz zinterpolowany wielomian.

\noindent \textbf{Opis rozwiązania} \\
Zaimplementować funkcję przy pomocy poprzednio zaimplemenotwanych algorytmów
oraz użyć wbudowane biblioteki do plotowania.
Algorytm najpierw generuje dane do użycia w poprzednio zaimplementowanym
algorytmie wyznaczania ilorazów różnicowych, a następnie generuje dane
porzebne do utworzenia wykresów funkcji oraz zinterpolowanego wielomianu.

\pagebreak

\noindent \textbf{Pseudokod}
\begin{algorithmic}
	\State \textbf{Input:} $f, a, b, n$
    \State $h \gets (b - a) / n$
    \State $xs \gets []$
    \State $ys \gets []$

    \For{$k $ \textbf{in} $0$ \textbf{to} $n$}

    \State $xs.push(a + k * h)$
    \State $ys.push(f(a + k * h))$

    \EndFor

    \State $ilorazy \gets ilorazyRoznicowe(xs, ys)$

    \State $n \gets n*100$
    \State $h \gets (b - a) / n$

    \State $printxs \gets []$
    \State $printys \gets []$

    \For{$k $ \textbf{in} $0$ \textbf{to} $n$}

    \State $printxs.push(a + k * h)$
    \State $printys.push(f(a + k * h))$

    \EndFor

    \State $polynomialValues \gets []$

    \For{$x $ \textbf{in} $printxs$}

    \State $polynomialValues.push(warNewton(xs, ilorazy, x))$

    \EndFor

    \State $plot(printxs, [printys, polynomialValues])$
\end{algorithmic}

\noindent \textbf{\large Zadanie 5}

\noindent \textbf{Opis problemu} \\
Przetestować funkcję z poprzedniego zadania na kilku przykładach.

\noindent \textbf{Opis rozwiązania} \\
Uruchomić funkcję używając jako danych wejściowych:

a) $e^x$, $[0,1]$, $n=5,10,15$

b) $x^2sin(x)$, $[-1,1]$, $n=5,10,15$

\pagebreak

\noindent \textbf{Rezultat}

\begin{center}
	\includegraphics[scale=0.6]{e^x500plot.png}

	\textbf{$e^x$ $n=5$}
\end{center}

\begin{center}
	\includegraphics[scale=0.6]{e^x1000plot.png}

	\textbf{$e^x$ $n=10$}
\end{center}

\begin{center}
	\includegraphics[scale=0.6]{e^x1500plot.png}

	\textbf{$e^x$ $n=15$}
\end{center}

\begin{center}
	\includegraphics[scale=0.6]{x^2*sinx500plot.png}

	\textbf{$x^2sin(x)$ $n=5$}
\end{center}

\begin{center}
	\includegraphics[scale=0.6]{x^2*sinx1000plot.png}

	\textbf{$x^2sin(x)$ $n=10$}
\end{center}

\begin{center}
	\includegraphics[scale=0.6]{x^2*sinx1500plot.png}

	\textbf{$x^2sin(x)$ $n=15$}
\end{center}

\pagebreak

\noindent \textbf{Wnioski} \\
Dla danych funkcji metoda interpolacji wielomianowej
bardzo dobrze jest w stanie przybliżyć \\ wartości funkcji.

\noindent \textbf{\large Zadanie 6}

\noindent \textbf{Opis problemu} \\
Przetestować funkcję z poprzedniego zadania na kilku przykładach żeby
zaprezentować zjawisko rozbieżności.

\noindent \textbf{Opis rozwiązania} \\
Uruchomić funkcję używając jako danych wejściowych:

a) $|x|$, $[-1,1]$, $n=5,10,15$

b) $\frac{1}{1+x^2}$, $[-5,5]$, $n=5,10,15$

\noindent \textbf{Rezultat}

\begin{center}
	\includegraphics[scale=0.6]{absx500plot.png}

	\textbf{$|x|$ $n=5$}
\end{center}

\begin{center}
	\includegraphics[scale=0.6]{absx1000plot.png}

	\textbf{$|x|$ $n=10$}
\end{center}

\begin{center}
	\includegraphics[scale=0.6]{absx1500plot.png}

	\textbf{$|x|$ $n=15$}
\end{center}

\begin{center}
	\includegraphics[scale=0.6]{1.01+x^2500plot.png}

	\textbf{$\frac{1}{1+x^2}$ $n=5$}
\end{center}

\begin{center}
	\includegraphics[scale=0.6]{1.01+x^21000plot.png}

	\textbf{$\frac{1}{1+x^2}$ $n=10$}
\end{center}

\begin{center}
	\includegraphics[scale=0.6]{1.01+x^21500plot.png}

	\textbf{$\frac{1}{1+x^2}$ $n=15$}
\end{center}

\noindent \textbf{Wnioski} \\
W przypadku funkcji z tego zadania widać, że interpolacja wielomianowa
dobrze aproksymuje funkcję tylko w okolicach środka przedziału, a przy
końcach zaczyna się bardzo mocno rozjeżdżać, w żaden sposób nie
przypominając danej funkcji.

\end{document}
